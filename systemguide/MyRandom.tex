% Copyright: CSCE240-001 - Fall 2016 - Group 7. All rights reserved.
% Date: 2016-12-1
%
% This chapter describes MyRandom class.
% This class is used to generate seudo-random numbers from a seed.


\chapter{MyRandom}
The MyRandom class is used to generate pseudo random numbrs from a seed provided.  The functions contained in MyRandom are used by this program to generate voter data.


\section{Constructor}
The MyRandom class contains overloaded constructor functions. The first instance uses the integer 1 as the seed value, while the second takes an integer parameter value that is used for the seed. 
If a seed value is provided, MyRandom will use that value, but if a seed value is not provided, MyRandom will use a seed value of 1.

\subsection{Seed Not Provided}
\begin{lstlisting}[language=C++]
 MyRandom::MyRandom() {
  seed_ = 1;
  generator_.seed(seed_);
}
\end{lstlisting}

\subsection{Seed Provided}
\begin{lstlisting}[language=C++]
 MyRandom::MyRandom(unsigned seed) {
  seed_ = seed;
  generator_.seed(seed_);
}
\end{lstlisting}
 
\section{General Functions}

\subsection{RandomExponentialInt}
\begin{description}
\item[Parameters] double lambda
\item[Returns] int r
\item[Usage] This function takes a double lambda as inpout, and returns an integer r.  The lambda is the is the lambda value of the exponentially distributed real numbers.  This function generates a double value based on the lambda constraint, rounds the double to the nearest integer, and returns this integer value.
This function is used in the Simulation Program to calculate the interval of voter arrival times.
\end{description}


\subsection{RandomNormal}
\begin{description}
\item[Parameters] double mean, double dev
\item[Returns] double r
\item[Usage] This function takes in doubles mean and dev for input, and generates a pseudo random number from a set with the provided mean and standard deviation.
\end{description}


\subsection{RandomUniformInt}
\begin{description}
\item[Parameters] int lower, int upper
\item[Returns] int r
\item[Usage] This function takes in two integer s as parameters, describing the maximum and minimum values that can be generated.  This function returns a pseudo-randomly generated value that falls between the upper and lower limits provided.
In the Simulation Program, this function is used to calculate the time taken for a voter to finish voting.  This function works well time taking to vote must be greater than zero, and less than the maximum service time specifed in the config file.
\end{description}



\section{Private Members}
\begin{description}
\item[unsigned int seed\_] The value given to start pseudo-random number generation.  This number can be provided when MyRandom is declared, or will be the default value 1.
\item[std::mt19937 generator\_] Pseudo-random number generator provided by the random library.
\end{description}


