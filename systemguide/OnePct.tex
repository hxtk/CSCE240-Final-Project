% Copyright: CSCE240-001 - Fall 2016 - Group 7. All rights reserved.
% Date: 2016-11-30
%
% This chapter describes the OnePct class.
% This class is an instance of a single voting precinct for use in the simulation.

\chapter{OnePct}

The \texttt{OnePct} class is where the bulk of the work for this program happens.

Much of the data stored are statistics that are intended only to be presented to the user. The program does not use them anywhere as of this version.

This class depends upon the following public members of \texttt{Configuration}:

\begin{center}
\begin{lstlisting}[language=C++]
int election_day_length_hours_;
int election_day_length_seconds_;
int time_to_vote_mean_seconds_;
int wait_time_minutes_that_is_too_long_;
int number_of_iterations_;
vector<int> actual_service_times_;
double arrival_zero_;
vector<double> arrival_fractions_;
\end{lstlisting}
\end{center}

\section{ToString} \label{ngrah-onepct-tostring}

This defines the canonical string form of a Precinct. \emph{Note: this is not the same as the canonical string form of a precinct simulation as described in Table~\ref{hxtk-canon-string-sim}.} The Canonical String Form has the following whitespace-delimited fields:

\begin{description}
\item[ID] The integer-valued ID of the precinct. This must be unique within its batch.
\item[Name] This is a human-readable text identifier for the precinct. \emph{Note: since fields are whitespace delimited, this cannot contain whitespace}.
\item[Turnout] This is a percentage (\%) of voters who voted out of those who were registered for this precinct.
\item[Voters] This is the number of people registered to vote at this precinct.
\item[Expected Voters]
\item[Voter Rate] This is the mean number of voters expected per hour.
\item[Booths] This is the number of voting booths being used by the precinct.
\item[\% Minority] This is the percentage of minority voters out of all voters.
\item[HH] This is the string literal ``HH'' and serves as the opening delimiter for the set of histogram breakpoints.
\item[Break Point(s)] These are the number(s) of voting booths for which a histogram should be printed. There may be up to three (3).
\item[HH] This is the string literal ``HH'' and serves as the corresponding closing delimiter.
\end{description}

\section{ReadData}

\begin{center}
\begin{tabular}{l r}
Access & \texttt{public} \\
Return Type & \texttt{void} \\
Arguments & \texttt{Scanner\&}
\end{tabular}
\end{center}

This function accepts an open Scanner pointing at a \emph{complete} standard serialization of an \texttt{OnePct} object. This is further detailed in \S\ref{hxtk-pct-file}. If the Scanner is not opened or is not pointing at a valid input string, the program will crash.

\emph{No member function may be called until this function has completed.}

\section{RunSimulationPct}

\begin{center}
\begin{tabular}{l r}
Access & \texttt{public} \\
Return Type & \texttt{void} \\ \hline
Arguments & \texttt{const Configuration\&}\\
          & \texttt{MyRandom\&}\\
          & \texttt{ofstream\&}
\end{tabular}
\end{center}

This function runs a series of simulations and analyzes their outcomes based on the data contained in the Configuration object.

Beginning with the minimum value that cannot be rejected trivially, it uses incrementally more voting booths in the simulation until none of the voters are waiting too long. This is done by creating a randomized set of voters with \texttt{CreateVoters} and passing off the processing of the queue to \texttt{RunSimulationPct2}.

Once that is complete, \texttt{DoStatistics} is executed and if this number of voting booths was specified as a breakpoint, a histogram is printed.

\section{CreateVoters}

\begin{center}
\begin{tabular}{l r}
Access & \texttt{private} \\
Return Type & \texttt{void} \\ \hline
Arguments & \texttt{const Configuration\&}\\
          & \texttt{MyRandom\&}\\
          & \texttt{ofstream\&}
\end{tabular}
\end{center}

This uses random elements from a log-normal data set which should be made available to the program as specified in \S\ref{hxtk-data-file}.

First a set of voters waiting at the door on poll open is created, and then an exponential random distribution is used to bring in a given rate of voters for each hour.

\section{RunSimulationPct2}

\begin{center}
\begin{tabular}{l r}
Access & \texttt{private} \\
Return Type & \texttt{void} \\
Arguments & \texttt{int}
\end{tabular}
\end{center}

This function accepts the number of voting booths which should be used by the precinct and attempts to process the queue. This program must be run after \texttt{CreateVoters}

\section{DoStatistics}

\begin{center}
\begin{tabular}{l r}
Access & \texttt{private} \\
Return Type & \texttt{int} \\ \hline
Arguments & \texttt{int}\\
          & \texttt{const Configuration\&}\\
          & \texttt{int}\\
          & \texttt{map$<$int, int$>$}\\
          & \texttt{ofstream\&}
\end{tabular}
\end{center}

This function performs analysis of the data from the simulations. It must be called \emph{after} \texttt{RunSimulationPct2}. This function calls \texttt{ComputeMeanAndDev} and is the only valid context in which to do so.

\iffalse % BLOCK COMMENT
\section{Member Variables}

\begin{description}
\item[int pct\_expected\_voters\_] Number of voters expected to vote at this precinct.
\item[int pct\_expected\_per\_hour\_] Number of voters expected to vote at this precinct over a one hour period.
\item[double pct\_minority\_] Percentage of a precinct's voters who identified as a minority.
\item[string pct\_name\_] Name of an individual precinct.
\item[int pct\_number\_] Number assigned as an identifier to an individual precinct.
\item[double pct\_turnout\_] Percentage of the number of expected voters who showed up to vote at a precinct.  This is not used in calculations.
\item[int pct\_stations\_] Number of voting stations at a single precinct.
\item[int pct\_num\_voters\_]  Total number of voters who voted in a precinct.
\item[double wait\_dev\_seconds\_] The standard deviation of the wait times of the voters in a precinct in seconds.
\item[double wait\_mean\_seconds\_] The mean wait time of the voters in a precinct in seconds.
\item[set$<$int$>$ stations\_to\_histo\_] Set containing the number of voting stations used in the simulation.  This is meant to be displayed as part of a histogram of the data.
\item[vector$<$int$>$ free\_stations\_] Vector containing the stations not currently in use.  This is used in the RunSimulationPct2 function.
\item[multimap$<$int, OneVoter$>$ voters\_backup\_] This is a map containing all of the voters created by the CreateVoters function.  This map is populated before the real work of the simulation begins.
\item[multimap$<$int, OneVoter$>$ voters\_done\_voting\_] This map contains the voters who have already voted in the simulation. Voters in voters\_voting\_ are moved here when they have finished voting.
\item[multimap$<$int, OneVoter$>$ voters\_pending\_] This map begins as a copy of voters\_backu\_ before any voting has occurred. Voters are removed from this map as they finish voting, and are added voters\_done\_voting\_.
\item[multimap$<$int, OneVoter$>$ voters\_voting\_] This map contains voters who are currently at a voting station. Once a voter finishes voting, they are moved to voters\_done\_voting\_.
\end{description}


\section{General Functions}

\subsection{ReadData}
\begin{description}
\item[Parameters] Scanner\& infile
\item[Returns] void
\item[Usage] ReadData is passed in a reference to a scanner as input.  The data read by the scanner is used to provide values for the member variables of an instance of OnePct.
\end{description}


\subsection{RunSimulationPct}
\begin{description}
\item[Parameters] const Configuration\& config, MyRandom\& random, ofstream\& out\_stream
\item[Returns] void
\item[Usage] RunSimulationPct does the real work when simulating a single voting precinct.  This function generates voters, simulates voting for a precinct, and collects and stores the data from the simulation.
\end{description}
% TO-DO 

\subsection{ToString} \label{ngrah-onepct-tostring}
\begin{description}
\item[Parameters] none
\item[Returns] string s
\item[Usage] ToString Formats the information collected during the voting simulation, as well as the expected voters and expected voters per hour, and stores it to a string s.  String s is returned.
\item[Format] pct\_number  pct\_name\_ pct\_turnout\_  pct\_num\_voters\_  pct\_expected\_voters\_  pct\_expected\_per\_hour\_  pct\_stations\_ pct\_minority\_  "HH" Stations\_to\_histo\_ "HH"
\item[Example] 1  XXX00100 20.20      10101     100     235  8   10.30 HH    0 HH
\end{description}

\subsection{ToStringVoterMap}
\begin{description}
\item[Parameters] string label, multimap$<$int, OneVoter$>$ themap
\item[Returns] string s
\item[Usage] Takes a map of instances of voters as input.  This function iterates through the map of voters, calling ToString for each, and storing the returned string into string s.  This function then returns string s.
\end{description}


\section{General Private Functions}

\subsection{CreateVoters}
\begin{description}
\item[Parameters] const Configuration\& config, MyRandom\& random, ofstream\& out\_stream
\item[Returns] void
\item[Usage] This function is called by RunSimulationPct.  This function uses input from the config file, number of expected voters, and the random number generator to generate all the instances of voters used to simulate voting in a single precinct.
\end{description}

\subsection{DoStatistics}
\begin{description}
\item[Parameters] int iteration, const Configuration\& config, int station\_count, map$<$int, int$>$\& map\_for\_histo, ofstream\& out\_stream
\item[Returns] toolongcount (Number of voters who waited for too long)
\item[Usage] This function is called by RunSimulation.pct to determine the mean and standard deviation of vote times, and the number of voters who waited for too long at a single precinct.
\end{description}

\subsection{ComputeMeanAndDev}
\begin{description}
\item[Parameters] none
\item[Returns] void
\item[Usage] This function is called by the DoStatistics function.  This function calculates the mean and standard deviation of the wait times of voters for a single precinct.
\end{description}

\subsection{RunSimulationPct2}
\begin{description}
\item[Parameters] int stations\_count
\item[Returns] void
\item[Usage] This function is called by the RunSimulationPct function.  This function takes the number of open voting stations as input, and simulates moving the line of waiting voters through the voting stations.
\end{description}

\fi % END BLOCK COMMENT
